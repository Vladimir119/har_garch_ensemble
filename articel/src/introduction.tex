% 1) Explain why predicting Bitcoin volatility is crucial for traders, investors, or \textbf{risk managers}. Discuss the unique challenges of predicting cryptocurrency volatility compared to traditional assets.

% 2) Highlight that GARCH is a standard model in such cases

% 3) Mention novelty, i.e., outperforming the naive approach
\par 

Volatility forecasting is a cornerstone of financial econometrics, especially in dynamic and rapidly evolving markets like cryptocurrencies. The importance of volatility in determining option prices, Value at Risk (VaR), and the Sharpe ratio underscores its relevance to investors and risk managers. However, the extreme fluctuations and unpredictability of cryptocurrency prices present unique modeling challenges. Among the tools available, the Heterogeneous Autoregressive (HAR) model, first introduced by \cite{corsi2009}, offers a robust framework for capturing time-varying volatility across different time horizons.

This study evaluates the performance of HAR-type models in predicting the realized volatility of cryptocurrency returns, comparing their accuracy to GARCH-type models \cite{bollerslev1987}, naive benchmarks, and various combinations of these approaches.

Building upon our previous research, we employ a comprehensive methodology that integrates distributional analysis and rolling-window prediction techniques to assess model efficacy under diverse market conditions.
